\chapter{Installation}

\section{Requirements}

\begin{flushleft}
\begin{description}
    \item[\href{http://www.python.org}{Python 2.5}]
        -- Although modu should run properly under Python 2.4,
        there are some memory release issues that can be a problem for long-
        running applications, such as web servers.
    \item[\href{http://www.zope.org/Products/ZopeInterface}{Zope Interface 3.3.0}]
        -- Zope.Interface is required by both Twisted
        and modu, but should be considered an essential part of any Python
        installation.
    \item[\href{http://www.twistedmatrix.com}{Twisted 2.5.0}]
        -- As a matter of personal preference, I run Twisted
        from the most recent SVN checkout, but the most recent stable release
        should be fine.
    \item[\href{http://www.mysql.com}{MySQL 5.0}]
        -- Modu is generally DB-agnostic, but at this time only
        MySQL 5 is supported.
    \item[\href{http://sourceforge.net/projects/mysql-python}{MySQLdb 1.2.2}]
        -- Earlier versions of MySQLdb have a bug in their
        unicode handling. Note that even if you're experimenting with alternate
        DB engines, there's a dependency on MySQLdb when generating queries.
\end{description}
\end{flushleft}

\section{Install Notes}

You should be able to install the items above in the order listed. There are
no special configuration details for the dependencies in general, but there
are a number of version- or platform-specific issues you might encounter.

Once the dependencies are all in place, you can uncompress a recent modu
tarball in some convenient location, which for now we'll call \emph{/usr/local/modu}.

Assuming root privileges, install the modu toolkit with:

\begin{verbatim}
    python setup.py install
\end{verbatim}

\subsection{Twisted Python on Mac OS X 10.5.1}

The stock Leopard install includes a partial install of Twisted in:
\begin{verbatim}
    /System/Library/Frameworks/Python.framework/Versions/2.5/Extras
\end{verbatim}
If you attempt to run Twisted out of the SVN module itself (say, for development
purposes), you can run into problems where the partial installed version preempts
the newly installed copy.

To fix this, you can place the following in a .pth file inside your site-packages directory:

\begin{verbatim}
import sys; sys.path = [s for s in sys.path 
                        if s != '/Library/Python/2.5/site-packages']
import sys; sys.path.insert(0, '/Library/Python/2.5/site-packages')
import sys; sys.path.insert(0, '/usr/local/dram/Twisted')
/usr/local/dram/modu
\end{verbatim}

The first line removes the site-packages directories (to prevent duplicates),
the next two lines insert the new paths at the top of the system path, and the
last line adds a directory to the end of the system path.

For more information, consult the \href{http://docs.python.org/lib/module-site.html}{Python documentation}.


\subsection{MySQLdb on Mac OS X 10.5.1}

There seems to be an issue with MySQLdb 1.2.2 (the Python MySQL driver) and
Mac OS X Leopard. The symptom is this error:

\begin{verbatim}
    In file included from /usr/local/mysql/include/mysql.h:43,
                     from _mysql.c:40:
    /usr/include/sys/types.h:92: error: duplicate 'unsigned'
    /usr/include/sys/types.h:92: error: two or more data types
                     in declaration specifiers
    error: command 'gcc' failed with exit status 1
\end{verbatim}

You'll need to edit \_mysql.c and comment out or remove the following lines:

\begin{verbatim}
    #ifndef uint
    #define uint unsigned int
    #endif
\end{verbatim}

Also, importing the MySQLdb module in Python may raise an exception about
a 'missing image'. If so, you'll also need to create this impossible-
looking symlink:

\begin{verbatim}
    ln -s /usr/local/mysql/lib /usr/local/mysql/lib/mysql
\end{verbatim}

